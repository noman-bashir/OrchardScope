To the best of our knowledge, there is no single IoT platform designed for precision agriculture applications that is \emph{simple}, \emph{extensible}, and \emph{open-source}. There have been various academic and commercial efforts that target specific precision agriculture applications, which are discussed next. 

% \noindent
% \textbf{IoT and Edge Computing:}
% The high level points: a lot of work on high level general problems that we save at different tiers, but the precision agriculture applications bring out unique aspects of that challenge. 
% that we can leverage to solve different problems that arise in different tiers of our solution, i.e. 
In the IoT and edge computing space, there has been relevant work on computation offloading to the edge~\cite{computation-offloading,reality-check-edge}, fault detection~\cite{fall-curve}, and energy-efficiency~\cite{wsn-pa}. To our knowledge no prior work provides one unfied solution that seeks to address these challenges in one cohesive system.

% \noindent
% \textbf{Precision Agriculture Architecture and Platforms:}
There has also been significant work in academia and industry on developing architecture and platforms for precision agriculture applications~\cite{vasisht2017farmbeats, smart-farming, smart-farm-arch, towards-smart-farming, energy-efficient-pa-system, precision-monitor-cabbage}. However, none of these solutions use a densely deployed network of sensors that enable high spatial resolution required by applications such as microclimate analysis. They also either completely rely on cloud, assume reliable network, or focus only on specific applications. Similarly, the precision agriculture solutions from industry are generally analytics oriented with no real-time support and are highly application-specific~\cite{see&spray, cropx, precisionhawk, fieldagent-sentera} and it is unknown how cost effective these solutions are for small farms. 

% common features: no densely deployed batteryless bottom tier that enables high spatial resolution, reliance on cloud, no edge , assume internet connectivity.

% Microsoft Research's FarmBeats~\cite{vasisht2017farmbeats} is an end-to-end IoT platform for data-driven agriculture with precision agriculture being one of its applications. The key problem with FarmBeats is the lack of flexibility in the architecture, i.e. requiring the presence of a farmer's home at the farm with a computer and internet connectivity. These assumptions may not be true for many farms, especially in the developing countries. Also, FarmBeats is primarily a data acquisition system and does not possess the capability to perform real-time inferences and control over actuators inside the farm, i.e. processing visual data in real-time to release the insecticide spray. 

% In addition, there has been various commercial efforts that focus on one or a few applications at best. For example, See \& Spray technology by Blue River focuses only on smart herbicide spraying~\cite{see&spray}, CropX platform is designed for smart irrigation and fertilizing~\cite{cropx}, and FieldAgent and PrecisionHawk systems focus on generating field maps to monitor crop growth and health~\cite{precisionhawk, fieldagent-sentera}. The heterogeneity of devices and proprietary technology used by different vendors will give rise to similar problems of interoperability and complexity faced by the Smart Home and IoT community today~\cite{home-os-challenges}. 

% An architecture for smart farming, layered architecture~\cite{smart-farm-arch}.
% an agricultural monitoring system for smart farming~\cite{smart-farming}.
% Towards smart farming~\cite{towards-smart-farming}. 
% Low-cost aerial~\cite{jain2019low}.
% experience of deploying always on farm~\cite{kapetanovic2017experiences}.
% energy-efficient PA system~\cite{energy-efficient-pa-system}.
% edge digital farm~\cite{edge-digital-farm}.
% precision monitoring cabbage and capsicum~\cite{precision-monitor-cabbage}.


